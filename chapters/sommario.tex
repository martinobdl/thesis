\begin{abstract}
Una delle sfide più importanti in finanza è quella di avere prestazioni migliori rispetto ad un approccio passivo agli investimenti. I movimenti casuali del mercato e la difficoltà nel predirne i catalizzatori rendono molto ostico surclassare il mercato, e quindi, in un ambito tanto arduo, tecniche progettate per tenere bassi i costi di transazione possono avere un impatto significativo sul guadagno finale. Questa tesi si concentra su tecniche di investimento basate su \emph{Online Portfolio Optimization} controllando i costi di transazione. Questo ambito si differenzia dal classico approccio poiché assume che i mercati abbiano un comportamento avversario, senza nessuna caratteristica stocastica, il che richiede quindi che tali tecniche ridistribuiscano di frequente il loro portfolio. Molti degli algoritmi in questo ambito non considerano i costi di transazione; mostreremo che quelli che hanno un bound sui costi lo fanno con assunzioni irrealistiche. Si propone l'uso di \emph{Online Gradient Descent} per trattare il problema dei costi di transazione in Online Portfolio Optimization. Mostreremo che questo assicura un bound sul guadagno, con costi, dell'ordine di $\mathcal O(\sqrt T)$, dove $T$ è l'orizzonte temporale. Inoltre mostreremo che questo algoritmo ha complessità computazionale dell'ordine di $\Theta(N)$, dove $N$ è il numero di azioni nel portfolio.
Alla fine verificheremo sperimentalmente le garanzie teoriche dell'algoritmo e che esso, quando testato su dati reali, provvede a guadagni comparabili agli altri algoritmi nello stato dell'arte.
\end{abstract}