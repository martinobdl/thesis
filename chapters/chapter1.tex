\chapter{Introduction}

Classical investment techniques for the portfolio management problem derive from the knowledge of the statistical distribution of the assets return. Then, once the statistical model has been chosen, the problem get solved by optimizing the expected value of the utility of some random variable (usually accounting for the trade-off between risk and return), that describes the value of the portfolio in some fixed time in the future. This line of thinking has been proposed and sustained by Markovitz, Samuelson, Fama ecc... %\cite{Fortune Formula}
, and it is now called Modern Portfolio Theory (MPT).

This approach is known to be very susceptible to the errors in the modelling of the random variable that model the asset return. In fact, it is known that the markets have a non stationary behaviour, which means that every statistical assumption is ephemeral and unreliable. %\cite{qualcuno} 
and they are usually referred to to backward looking, i.e. that they optimize 

A different approach has been originated from the fields of information theory at the Bell Labs in the 1950, from the works of Shannon, Kelly and Cover. This methods were first included in the classical portfolio theory framework, under the name of Capital Growth Theory (CGT) \cite{hakansson1995capital} and then got included in the machine learning literature under the framework of Online Game Playing. Only recently this field has been taken into the Online Optimization This formulation has very interesting properties such as stability in a game theory fashion (equilibrium) and robustness versus adversarial manipulation.

One of the strongest points in favor of this techniques are the strong theoretical guarantees that algorithms developed under this framework can give. This guarantees come from the game theory concept of Regret, which is a form of dissatisfaction originated from having taken an action, instead of another action.

Principal in this thesis will be the extension of the modelling of the financial applications of this methodologies to the presence of transaction costs and to provide strong theoretical assurance even in the presence of transaction costs. In fact in many financial situations, transaction costs are not modelled and 
