\chapter{Problem Formulation}\label{ch:transaction_costs}

In this section we will extend the Online Portfolio Optimization model to consider transaction costs. The resulting framework will be central in our contributions.  
Indeed, the importance of the trading mechanism is usually not taken into account in the models of Online Portfolio Optimization, notably, the most important aspect left out of the analysis is transaction costs. Including transaction costs into the Online Portfolio Optimization model is non-trivial and complex. The reason why transaction costs are more difficult to include into the model is that the inclusion of transaction costs change significantly the loss function and, as we saw, the theoretical guarantees of the algorithms do relay heavily on very strict conditions on the loss function, such as convexity and exp-concavity.
Note that an algorithm that guarantees sub-linear regret without transaction costs is not guaranteed to have sub-linear regret in the more realistic scenario in which trading costs are present. 

Very few works include transaction costs in the Online Portfolio Optimization model. 
There exists a wide variety of heuristic methods that tried to overcome this problem~\cite{li2018transaction,yang2018reversion}, but they do not provide any guarantee on the regret in the presence of transaction costs. 
To the best of our knowledge, there are only two studies that analyze transaction costs theoretically: Universal Portfolio with Costs (U$_C$P)~\cite{blum1999universal}, and Online Lazy Updates (OLU)~\cite{das2013online}, but only OLU gives an algorithm to implement. We will present the algorithm designed in these works in Chapter \ref{ch:algos}. The principal contribution of this thesis is to give an algorithm that has sub-linear regret in the Online Portfolio Optimization problem with transaction costs.

\section{Online Portfolio Optimization with Transaction Costs}

Transaction costs are notably difficult to model or even define. In order to model trading costs correctly, one would have to take into account many aspects of the trading mechanism and explore the mechanism of trading in its minutiae; this field of research is called \emph{market micro-structure}. Great starting references can be found in \cite{harris2003trading} and \cite{o1997market}, in which the authors explore the practical implementation of a trade and its analytical formulation, respectively. 

Conversely, we model transaction costs as in \cite{das2013online}. We showed how the model used by the authors can be recovered as an approximation of the proportional transaction costs model, that will be discussed in detail in the next section. Finally, we checked the consistency of the empirical results in the used model and the model used in this thesis.

We also used this model for transaction costs to define a new concept of regret to be used in the framework of Online Portfolio Optimization.

Following the approach previously used in the Online Portfolio Optimization literature~\cite{blum1999universal}, we use an approximation of the real transaction costs considering them proportional to the difference in portfolio allocation.
Formally, the transaction costs, at round $t$, are implicitly determined by the solution of the following equation:

\begin{align}\label{eq:real_tc_big}
W_{t-1}&=\tilde W_{t-1}+\gamma_s\sum\limits_{i=1}^N\left(\frac{x_{i,t-1}y_{i,t-1}}{\langle \mathbf x_{t-1},\mathbf y_{t-1}\rangle}-x_{i,t}\tilde W_{t-1}\right)^+\\\nonumber
&+\gamma_b\sum\limits_{i=1}^N\left(x_{i,t}\tilde W_{t-1}-\frac{x_{i,t-1}y_{i,t-1}}{\langle \mathbf x_{t-1},\mathbf y_{t-1}\rangle}\right)^+,
\end{align}

where $\gamma_s,\gamma_b>0$ are the proportional transaction fees for selling and buying respectively, $W_{t-1}$ is the wealth before the trading costs are taken into account, $\tilde W_{t-1}$ is the wealth remaining after the trading costs, and $(x)^+$ is defined as the positive part of $x$ as $(x)^+:=\max(x,0)$.
This model for transaction costs is called \emph{proportional transaction costs}. There is no work in the scientific literature able to bound theoretically the wealth of an online learning algorithm when this model of costs is adopted.

If we assume that in Equation \eqref{eq:real_tc_big} we have $\gamma=\gamma_s=\gamma_b>0$ equal and fixed both throughout the investment horizon $T$ and for buying ans selling and defining $\alpha_t:=\frac{\tilde W_{t-1}}{W_{t-1}}$, we can rewrite Equation \eqref{eq:real_tc_big} as:

\begin{equation}\label{eq:real_tc}
   \alpha_t = 1 - \gamma ||\mathbf{x}'_{t-1}-\mathbf{x}_t \alpha_{t} ||_1,
\end{equation}

where $\mathbf{x'}_{t-1} = \frac{\mathbf{x}_{t-1} \otimes \mathbf{y}_{t-1} }{\langle \mathbf{x}_{t-1}, \mathbf{y}_{t-1} \rangle }$ is the portfolio composition after the market movement $\mathbf{y}_{t-1}$.  With $ \mathbf{a} \otimes \mathbf{b}$ we denote the element-wise product between the two vectors $\mathbf{a}$ and $\mathbf{b}$.

With this model, the wealth that takes into account transaction costs
can be written as:

\begin{equation} \label{eq:realwealth}
    \tilde W_T = \prod\limits_{t=1}^T \alpha_t\langle \mathbf{x}_t, \mathbf{y}_t \rangle,
\end{equation}

where $\alpha_t$ is the solution of Equation \eqref{eq:real_tc}. We simplify further Equation \eqref{eq:real_tc} to avoid having to work with a non-linear equation. Indeed, if we assume that the components of $\mathbf y_t$ are small, we can assume $\mathbf x'_t\approx \mathbf x_t$ and $\alpha_t\mathbf x_t\approx\mathbf x_t$. Therefore, the ratio of the wealth remaining after the trading costs can be approximated by:

\begin{equation}\label{eq:fake_tc}
\alpha_t\approx1-\gamma||\mathbf x_t-\mathbf x_{t-1}||_1.
\end{equation}

We will now define a new concept of regret for the Online Portfolio Optimization, compared to the one originated from the log-loss of Section \ref{sec:from_horse_to_ptf}. 
Formally, using the approximation of the cost provided in Equation (4.4) we have:

\begin{align}
    \log(\tilde W_T)&=\log\left(\prod\limits_{t=1}^T  \langle \mathbf{x}_t, \mathbf{y}_t \alpha_t\rangle\right) \\ 
    & = \log(W_T)+\log\left(\prod\limits_{t=1}^T \alpha_t\right) \\ 
    & \approx \log(W_T) - \sum\limits_{t=1}^T\gamma||\mathbf{x}_t-\mathbf{x}_{t-1}||_1. \label{eq:expansion}
\end{align}

The approximation in Equation \eqref{eq:expansion} is because $\gamma\ll1$ and $\log(1-x)\approx-x$.

Hence we can define the quantity:
\begin{equation}\label{eq:l1_wealth}
W_T^C:=W_T\exp\left\{-\gamma\sum\limits_{t=1}^T||\mathbf{x}_t-\mathbf{x}_{t-1}||_1\right\},
\end{equation} 
which is the wealth obtained by an algorithm assuming transaction costs given by Equation \eqref{eq:fake_tc}.


Note that by using Equation \eqref{eq:fake_tc} we have that the BCRP pays zero transaction costs, and this observation justifies further the use of the following definitions:

\begin{definition}(Regret on the costs)\label{def:regret_on_the_costs}
For the Online Portfolio Optimization with transaction costs problem we define:
\begin{equation}
C_T:=\gamma\sum\limits_{t=1}^T||\mathbf x_t-\mathbf x_{t-1}||_1,
\end{equation}
as the regret on the costs paid by a learner predicting the sequence $\mathbf x_1,\ldots,\mathbf x_T$ of portfolio vectors.
\end{definition}

Hence, under this model, the quantity $C_T$ can be interpreted either as the costs paid by the learner or as the gap between the costs paid by the learner and the best expert in the class, which is zero.

Using the previous definition we are now able to introduce the concept of total regret.

\begin{definition}(Total Regret)\label{def:totoal_regret}
For the Online Portfolio Optimization with transaction costs problem we define for an online learning algorithm $\mathcal A$: 
\begin{equation}
R_T^C(\mathcal A):=R_T(\mathcal A)+C_T(\mathcal A),
\end{equation}
where $R_T(\mathcal A)$ is defined as the log-loss regret introduced in Section \ref{sec:from_horse_to_ptf} and $C_T(\mathcal A)$ is the regret on the costs defined in Definition \ref{def:regret_on_the_costs}.
\end{definition}

We are mainly interested in algorithms that bound the total regret $R_T^C$, as we believe that this line of research might potentially start to close the bridge between theory and practice in the Online Portfolio Optimization framework.


\section{Related Works in Online Learning and Switching Costs}
Finally, it is worth noting that the problem of dealing with transaction costs has also been tackled in sequential decision-making settings similar to the Online Portfolio Optimization one, \emph{i.e.}, in the expert and bandit learning~\cite{li2018online,cesa2013online,trovo2016budgeted} and the Metrical Task Systems literature~\cite{lin2012online}, where the notion of regret has been extended to include the cost of changing the prediction of the algorithm over time. These algorithms cannot be applied directly to the problem of Online Portfolio Optimization because their setting is notably different. 
For example in \cite{li2018online} the authors are concerned with Online Learning in applications where the outcomes are very predictable (\emph{e.g.}, energy consumption) and hence they assume to have knowledge of the $n$ future outcomes. This assumption is clearly not met in general in financial assets.
On the other hand, the Multi Arm Bandit framework assumes that the learning agent has knowledge of the loss function only for the action taken at the previous step, and cannot compute the loss associated with the other actions. Nonetheless, in \cite{ito2018regret} the authors suggested that the partial observability of the Multi Armed Bandit framework could be used to model illiquid assets such as the real estate market.
