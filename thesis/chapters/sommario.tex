\begin{poliabstract}{Sommario}
Una delle sfide più importanti in finanza è quella di avere prestazioni migliori rispetto ad un approccio passivo agli investimenti. I movimenti casuali del mercato e la difficoltà nel predirne i catalizzatori rendono molto complesso battere il mercato, e quindi, in un ambito tanto competitivo, tecniche progettate per tenere bassi i costi di transazione possono avere un impatto significativo sul guadagno finale. Questa tesi si concentra su tecniche di investimento basate su \emph{Online Portfolio Optimization} controllando i costi di transazione. Questo ambito si differenzia dal classico approccio poiché assume che i mercati abbiano un comportamento avversario, ossia non richiede delle assunzioni sul modello stocastico del processo, il che richiede quindi che tali tecniche ridistribuiscano di frequente il loro portfolio. Molti degli algoritmi in questo ambito non considerano i costi di transazione; mostreremo che quelli che hanno delle garanzie teoriche sui costi lo fanno con assunzioni irrealistiche. Si propone l'uso di \emph{Online Gradient Descent} per trattare il problema dei costi di transazione in Online Portfolio Optimization. Mostreremo che questo algoritmo assicura un regret sul guadagno con costi dell'ordine di $\mathcal O(\sqrt T)$, dove $T$ è l'orizzonte temporale. Inoltre mostreremo che questo algoritmo ha complessità computazionale dell'ordine di $\Theta(N)$, dove $N$ è il numero di azioni nel portfolio.
Infine verificheremo sperimentalmente le garanzie teoriche dell'algoritmo e che esso, quando testato su dati reali, provvede a guadagni comparabili agli altri algoritmi nello stato dell'arte. Abbiamo testato gli algoritmi scelti su tre datasets usati comunemente in letteratura e su un dataset raccolto per questo lavoro. Su tutti i dataset otteniamo guadagni medi del portfolio comparabili agli altri algoritmi per piccoli valori del tasso di transazione (guadagno annualizzato di tra 8\% e 15\%, approssimativamente), e guadagni più grandi, rispetto agli altri algoritmi, per quasi tutti i dataset utilizzati. 

\end{poliabstract}