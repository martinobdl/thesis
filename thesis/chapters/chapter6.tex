\chapter{Online Gradient Descent for Online Portfolio Optimization with Transaction Costs}\label{ch:OGD}

The Online Gradient Descent (OGD) algorithm is one of the first algorithms developed in the field of Online Convex Optimization \cite{zinkevich2003online}. We extended its use to the Online Portfolio Optimization framework and proved that the OGD algorithm has many interesting properties, among which a bound on the total regret $R_T^C$.

\begin{algorithm}
    \caption{OGD in Online Portfolio Optimization with Transaction Costs}
    \label{alg:OGD_in_OPO}
    \begin{algorithmic}[1]
    \REQUIRE learning rate sequence $\{\eta_1, \ldots, \eta_T\}$  \nonumber
    \STATE Set $\mathbf{x}_1 \gets \frac{1}{N} \mathbf{1}$ \label{line:init_OGD}
    \FOR {$t \in \{ 1, \ldots, T \}$}
    \STATE $\mathbf{z}_{t+1} \gets \mathbf{x}_{t}+ \eta_t \frac{\mathbf{y}_t}{\langle \mathbf{y}_t,\mathbf{x}_t \rangle}$ \label{line:update_OGD}
    \STATE Select Portfolio $\mathbf x_{t+1}=\Pi_{\Delta_{N-1}}(\mathbf z_t)$ \label{line:line_projection_OGD}
    \STATE Observe $\mathbf{y}_{t+1}$ from the market \label{line:out_OGD}
    \STATE Get wealth $\log( \langle \mathbf{y}_{t+1},\mathbf{x}_{t+1} \rangle) - \gamma|| \mathbf{x}_{t+1} - \mathbf{x}_{t} ||_1$ \label{line:wealth}
    \ENDFOR
    \end{algorithmic}
\end{algorithm}

\section{Using OGD for Portfolio Optimization} \label{sec:analysis}

This section describes the adaptation of the OGD algorithm to the Online Portfolio Optimization framework and provides a theoretical analysis of such an algorithm in the presence of transaction costs.

\subsection{The OGD Algorithm}
\label{sec:OGD}
The definition of the OGD update rule for a generic convex loss function $f_t(\mathbf{x}_t)$ over a generic convex set $\mathcal D$ is the following:
\begin{equation}\label{eq:OGD_general}
  \mathbf{x}_{t+1} = \Pi_{\mathcal D} \left( \mathbf{x}_t - \eta_t \nabla f_t(\mathbf{x}_t) \right),
\end{equation}
where $\Pi_{\mathcal D} (y) := \arginf\limits_{x \in \mathcal D } || y - x ||_2^2$ is the standard projection of the vector $y$ onto $\mathcal D$, $\eta_t > 0$ is the learning rate at round $t$.
This procedure is also reported in Figure \ref{fig:OGD}, where the point $\mathbf x_t$ is updated to $\mathbf z_t$, by the gradient of the loss, and then it gets projected onto the convex set $\mathcal D$ into the point $\mathbf x_t$.

\begin{figure}[ht!]
\centering
\begin{pspicture}(0,-2.35)(6.9,2.35)
\psellipse[linecolor=black, linewidth=0.04, dimen=outer](3.45,-0.4)(3.45,1.95)
\rput[bl](0.4,-0.55){$\mathcal D$}
\psdots[linecolor=black, dotsize=0.14078125](4.9,1.35)
\psline[linecolor=black, linewidth=0.04, arrowsize=0.05291667cm 2.0,arrowlength=1.4,arrowinset=0.0]{->}(4.9,1.35)(3.3,0.55)
\rput[bl](2.3,0.85){$\eta\nabla f_t(\mathbf x_t)$}
\rput[bl](4.7,1.45){$\mathbf x_t$}
\psline[linecolor=black, linewidth=0.04, arrowsize=0.05291667cm 2.0,arrowlength=1.4,arrowinset=0.0]{->}(4.9,1.35)(6.411111,2.15)
\rput[bl](6.6,2.15){$\mathbf z_t$}
\psdots[linecolor=black, dotsize=0.15374984](6.4,2.15)
\psline[linecolor=black, linewidth=0.04, arrowsize=0.05291667cm 2.0,arrowlength=1.4,arrowinset=0.0]{->}(6.3636365,2.0954545)(5.7272725,1.0045455)
\rput[bl](5.2363634,0.6227273){$\mathbf x_{t+1}$}
\end{pspicture}
\caption{Online Gradient Descent.}
\label{fig:OGD}
\end{figure}

Recalling that in the Online Portfolio Optimization framework the function to be minimized is the loss $f_t(\mathbf{x}_t) = -\log (\langle \mathbf{x}_t, \mathbf{y}_t \rangle )$, the portfolio update rule becomes:
\begin{equation} \label{eq:OGD_port}
   \mathbf{x}_{t+1}= \Pi_{\Delta_{N-1}}\left( \mathbf{x}_t+\eta_t \frac{\mathbf{y}_t}{\langle \mathbf{x}_t, \mathbf{y}_t \rangle}\right).
\end{equation}
The pseudo-code corresponding to the OGD algorithm is presented in Algorithm~\ref{alg:OGD_in_OPO}. 
The algorithm starts with a portfolio $\mathbf{x}_1$ equally allocated among the $N$ available assets (Line~\ref{line:init_OGD}).
Then, for each round $t \in \{ 1, \ldots, T \}$ it rebalances the assets according to Equation~\eqref{eq:OGD_port}, observes the market outcomes $\mathbf{y}_{t+1}$ (Line~\ref{line:out}), and gains a per-round wealth, including costs, of $\log(\langle \mathbf{y}_{t+1},\mathbf{x}_{t+1} \rangle) - \gamma|| \mathbf{x}_{t+1} - \mathbf{x}_{t} ||_1$ (Line~\ref{line:wealth}). The projection in Line \ref{line:line_projection_OGD}, can be implemented very efficiently as we will discuss in Section \ref{sec:implementation} with Algorithm \ref{alg:projection}.

Note that OGD is an instance of the OMD algorithm described in Section \ref{sec:OMD}, with $\psi(\mathbf x)=||\mathbf x||_2^2$. Indeed the general update Equation \eqref{eq:OGD_general} is equivalent to:

\begin{align}
	\mathbf x_{t+1}&=\arginf\limits_{\mathbf x\in\Delta}||\mathbf x-\mathbf x_t+\eta_t\nabla f_t(\mathbf x_t)||_2^2\\
	&=\arginf\limits_{\mathbf x\in\Delta}\left(||\mathbf x-\mathbf x_t||_2^2+\eta_t^2||\nabla f_t(\mathbf x_t)||_2^2+2\langle\nabla f_t(\mathbf x_t),\mathbf x-\mathbf x_t\rangle\right).
\end{align}

Moreover the following lemma is paramount to prove the regret bound for OGD. This lemma establishes the non expansiveness of the projection operator $\Pi_\Delta$:

\begin{lemma}(Generalized Pythagorean Theorem.)\label{lemma:non_expansive}
Let $\mathcal D\in\mathbb R^N$ a convex set, and $A\in\mathbb R^{N\times N}$ a semi-positive defined matrix. Then, for any point $\mathbf x\in\mathbb R^N$, we have:

\begin{equation}
\langle\mathbf x-\mathbf x_0,A(\mathbf x-\mathbf x_0) \rangle\ge\langle\mathbf z-\mathbf x_0,A(\mathbf z-\mathbf x_0)\rangle, \forall\mathbf x_0\in\mathcal D,
\end{equation}
where $\mathbf z=\Pi_{\mathcal D}^A(\mathbf x)=\arginf\limits_{\mathbf y\in\mathcal D}\langle\mathbf y-\mathbf x,A(\mathbf y-\mathbf x)\rangle$.
\end{lemma}

In the case of $A=\mathbb I_{N\times N}$, being $\mathbb I_{N\times N}$ the identity matrix in $\mathbb R^{N\times N}$, we have that $||\Pi_\mathcal D(\mathbf x)-\mathbf x_0||_2^2\le||\mathbf x-\mathbf x_0||_2^2$. Hence, the operator $\Pi_\Delta:\mathbb R^N\to\mathcal D$ is non-expansive. In Figure \ref{fig:pitagora} we show that the projection $\Pi_\mathcal D^A(\mathbf x)$ is the closest (in terms of the metric induced by the matrix $A$) to any point $\mathbf x_0\in\mathcal D$.

\begin{figure}[ht!]
\centering
\begin{pspicture}(0,-2.6668491)(5.5673485,2.6668491)
\psbezier[linecolor=black, linewidth=0.04](0.4,-1.223151)(0.7222638,-0.27650103)(3.9077222,2.3008838)(4.9,2.176849060058594)(5.8922777,2.0528142)(5.522264,-0.876501)(5.2,-1.823151)(4.877736,-2.769801)(4.292278,-2.6471856)(3.3,-2.523151)(2.307722,-2.3991163)(0.0777362,-2.1698008)(0.4,-1.223151)
\psdots[linecolor=black, dotsize=0.16](4.9,-0.82315093)
\psdots[linecolor=black, dotsize=0.16](4.1,1.9768491)
\psdots[linecolor=black, dotsize=0.16](0.4,1.376849)
\rput[bl](0.0,0.9768491){$\mathbf x$}
\rput[bl](3.6,2.276849){$\Pi_{\mathcal D}^A(\mathbf x)$}
\rput[bl](4.6,-1.223151){$\mathbf x_0$}
\psline[linecolor=black, linewidth=0.016](0.4,1.376849)(4.1,1.9768491)
\psline[linecolor=black, linewidth=0.016](4.1,1.9768491)(4.9,-0.82315093)
\psline[linecolor=black, linewidth=0.016](0.4,1.376849)(4.9,-0.82315093)
\rput[bl](2.2,-1.723151){$\mathcal D$}
\end{pspicture}
\caption{Generalized Pythagorean Theorem.}
\label{fig:pitagora}
\end{figure}

\section{Regret Analysis}

In this section we will analyze both the regret and the total regret of the OGD algorithm in Online Portfolio Optimization. Indeed, we are able to recover sub-linear regret in both cases, without any assumption on the transaction rate parameter.

\subsection{OGD Regret on the Wealth}
We recall that, for a generic convex function $f_t(x)$, it has been shown in~\cite{belmega2018online} that $
R_T(OGD) = \mathcal{O}(\sqrt{T})$ if the loss function $f_t(x)$ is convex, as in our case.
We follow the proof in~\cite{zinkevich2003online} to derive the specific result for the regret of OGD in the Online Portfolio Optimization framework:
\begin{theorem}\label{th:convex_regret_OGD}
    If Assumption~\ref{ass:nojunk} holds, the OGD algorithm with $\eta_t = \frac{K}{\sqrt{t}}$, $\forall K \in \mathbb{R}^+$ has a regret on the wealth of:
    \begin{equation*}
        R_T(OGD) \leq \left( \frac{1}{K} + \frac{N K \epsilon_u^2}{\epsilon_l^2} \right) \sqrt{T}.
    \end{equation*}
\end{theorem}
\begin{proof}
Notice that the $L_2$ diameter of a simplex $\Delta_{N-1}$ is $D = \sqrt{2}$ for any $N$ and that, under Assumption~\ref{ass:nojunk}, it is possible to bound the gradient of the loss as follows:
\begin{equation} \label{eq:bounded_gradient}
    ||\nabla \log (\langle \mathbf{x}_t, \mathbf{y}_t \rangle)||_2 \leq \frac{\epsilon_u \sqrt{N}}{\epsilon_l}:=G_2.
\end{equation}
Given the update in Equation~\eqref{eq:OGD_port} for the OGD algorithm, we have:
\begin{align}
     ||\mathbf{x}_{t+1} - \mathbf{x}^*||_2^2 &= ||\Pi_{\Delta_{N-1}}(\mathbf{x}_t + \eta_t \nabla \log(\langle \mathbf{x}_t, \mathbf{y}_t \rangle)) - \mathbf{x}^*||_2^2 \nonumber\\
    \leq & ||\mathbf{x}^* - \mathbf{x}_t||_2^2-2\eta_t\langle \mathbf{x}_t-\mathbf{x}^*,\nabla \log( \langle \mathbf{x}_t, \mathbf{y}_t \rangle) \rangle \nonumber \\ 
    & + \eta_t^2||\nabla \log( \langle \mathbf{x}_t, \mathbf{y}_t \rangle)||_2^2, \label{eq:magic}
\end{align}
where we used the fact that the projection operator $\Pi_{\Delta_{N-1}}(\cdot)$ is non-expansive (Lemma \ref{lemma:non_expansive}).
Rearranging the terms, we have:
\begin{align*}
    \langle \mathbf{x}^*-\mathbf{x}_t, \nabla \log (\langle \mathbf{x}_t, \mathbf{y}_t \rangle)\rangle\le \frac{1}{2\eta_t}\left( ||\mathbf{x}_t-\mathbf{x}^*||_2^2-||\mathbf{x}_{t+1}-\mathbf{x}^*||_2^2\right) + \frac{\eta_t}{2}G_2^2.
\end{align*}

Using the above inequality and the convexity of the logarithm, we bound the regret $R_T(OGD)$ as follows:
\begin{align*}
     R_T (ODG) &= \sum\limits_{t=1}^T\log (\langle \mathbf{x}^*, \mathbf{y}_t \rangle)-\log (\langle \mathbf{x}_t, \mathbf{y}_t \rangle) \\
    & \le\sum\limits_{t=1}^T\langle\mathbf{x}^*-\mathbf{x}_t,\nabla \log (\langle \mathbf{x}_t, \mathbf{y}_t \rangle)\rangle\\
    &\le\sum\limits_{t=1}^T\left[\frac{1}{2\eta_t}\left(||\mathbf{x}_t-\mathbf{x}^*||_2^2-||\mathbf{x}_{t+1}-\mathbf{x}^*||_2^2\right)+\frac{\eta_t}{2}G_2^2\right]\\
    &\le\frac{1}{2\eta_1}||\mathbf x^*-\mathbf x_1||_2^2-\frac{1}{2\eta_T}||\mathbf x^*-\mathbf x_{T+1}||_2^2+\sum\limits_{t=2}^T\frac{1}{2\eta_t}||\mathbf x_{t}-\mathbf x^*||_2^2\\
    &-\sum\limits_{t=1}^{T-1}\frac{1}{2\eta_t}||\mathbf x_{t+1}-\mathbf x^*||_2^2+\sum\limits_{t=1}^T\frac{\eta_t}{2}G_2^2\\
    &\le \frac{D^2}{2\eta_1}+\frac{D^2}{2}\sum\limits_{t=2}^{T}\left(\frac{1}{\eta_t}-\frac{1}{\eta_{t-1}}\right)+\sum\limits_{t=1}^T\frac{\eta_t G_2^2}{2}\\
    &=\frac{D^2}{2\eta_T}+\sum\limits_{t=1}^T\frac{\eta_tG_2^2}{2} \le \left(\frac{D^2}{2K}+G_2^2K\right)\sqrt{T},
\end{align*}
where, for the last inequality, we used that $\sum_{t=1}^T \frac{1}{\sqrt{t}} \leq 2 \sqrt{T}$.
By plugging the expression of the $L_2$ diameter $D$ and the $L_2$ bound on the gradient $G_2$, we conclude the proof.
\end{proof}

\subsection{OGD Regret on the Costs}
In the following theorem, using techniques similar to the ones in~\cite{andrew2013tale}, we bound the transaction costs $C_T(OGD)$ of the OGD algorithm in the Online Portfolio Optimization framework:
\begin{theorem}\label{th:tc_ogd}
    If Assumption~\ref{ass:nojunk} holds, the OGD algorithm with $\eta_t = \frac{K}{\sqrt{t}}$,  $\forall K \in \mathbb{R}^+$ has a regret on the costs of:
    \begin{equation*}
        C_T(OGD) \leq \frac{2 N K \gamma \epsilon_u}{\epsilon_l} \sqrt{T}.
    \end{equation*}
\end{theorem}

\begin{proof}
Recall that, in this setting, the regret on the costs $C_T(OGD)$ is equivalent to the sum of the costs incurred by the OGD algorithm, since the best CRP incurs in no costs.
Therefore, we have:
\begin{align}
    C_T(OGD) & = \gamma \sum\limits_{t=1}^{T-1} ||\mathbf{x}_{t+1} - \mathbf{x}_{t}||_1 \\
    & \leq \gamma\sum\limits_{t=1}^{T-1} \sqrt{N} ||\mathbf{x}_{t+1} - \mathbf{x}_{t}||_2 \label{eq:cost1}\\
    & \leq \gamma\sum\limits_{t=1}^{T-1}\sqrt{N}|| \eta_t \nabla \log( \langle \mathbf{x}_t, \mathbf{y}_t \rangle)||_2 \label{eq:cost2}\\
    &\leq \gamma\sqrt{N} G_2 \sum\limits_{t=1}^{T-1} \eta_t \label{eq:cost3}\\
    & \leq 2\gamma G_2K\sqrt{NT}, \label{eq:cost4}
\end{align}
where we used the equivalence of the norms in $\mathbb{R}^N$ for the inequality in Equation~\eqref{eq:cost1}, the fact that the projection operator $\Pi_{\Delta}(\cdot)$ is non-expansive and the update formula for OGD to derive Equation~\eqref{eq:cost2}, and the fact that the gradient of the loss is bounded by $G_2$ in Equation~\eqref{eq:cost3}. 
Finally, we conclude the proof by substituting the bound on the gradient in Equation~\eqref{eq:bounded_gradient} into Equation~\eqref{eq:cost4}.
\end{proof}

\subsection{Total Regret}
Summarizing the bounds derived in Theorems~\ref{th:convex_regret_OGD} and~\ref{th:tc_ogd}, we obtain the following:
\begin{theorem} \label{thm:total_regret}
    If Assumption~\ref{ass:nojunk} holds, the OGD algorithm with $\eta_t = \frac{K}{\sqrt{t}}$,  $\forall K \in \mathbb{R}^+$ has a total regret of:
    \begin{align*}
        & R_T^C(OGD) \le \left[ \frac{1}{K} + \frac{NK\epsilon_u}{\epsilon_l}\left( \frac{\epsilon_u}{\epsilon_l} + 2 \gamma \right) \right] \sqrt{T}. %\label{eq: OGD_regret_plu_costs_bound}
    \end{align*}
\end{theorem}
If the investment horizon $T$ is known in advance, the learning rate $\eta_t$ can be tuned to obtain a slightly better upper bound on the total regret:
\begin{corollary}\label{cor:OGD_T_known}
   If Assumption~\ref{ass:nojunk} holds, the OGD algorithm with $\eta_t=\frac{K}{\sqrt{T}}$, $\forall K \in \mathbb{R}^+$ has a total regret of:
    \begin{align}\label{eq:OGD_T_known}
    R_T^C(OGD) &\leq \left( \frac{1}{K} + \frac{NK \epsilon_u^2}{2 \epsilon_l^2} \right) \sqrt{T} + 2 \gamma  \frac{\epsilon_u}{\epsilon_l} \sqrt{T}.
    \end{align}
\end{corollary}

Finally, knowing the value of $\epsilon_l$ and $\epsilon_u$ in Assumption \ref{ass:nojunk}, the parameter $K$ can be chosen to minimize the bound in Theorem~\ref{thm:total_regret}, giving the following result:

\begin{corollary} \label{cor:optimal_bound}
    If Assumption~\ref{ass:nojunk} holds, the OGD algorithm with $\eta_t = \frac{1}{\sqrt{t}} \left[ \frac{N \epsilon_u}{\epsilon_l} \left( \frac{\epsilon_u}{\epsilon_l} + 2 \gamma \right) \right]^{-\frac{1}{2}}$ has a total regret of:
    \begin{equation*}
        R_T^C(OGD) \leq 2\sqrt{\frac{N \epsilon_u}{\epsilon_l}\left( \frac{\epsilon_u}{\epsilon_l} + 2 \gamma \right) T}.
    \end{equation*}
\end{corollary}

In what follows, we compare the theoretical guarantees of OGD in terms of computational complexity and regret with OLU and U$_C$P, the only algorithms that provide upper bounds to total regret.

\section{Implementation of the Online Gradient Descent Algorithm}\label{sec:implementation}

The OGD algorithm can be implemented very efficiently, indeed all computations of Algorithm \ref{alg:OGD_in_OPO} are trivial and lightweight, except for the projection operator $\Pi_{\Delta_{N-1}}$ onto the simplex $\Delta_{N-1}$. In \cite{duchi2008efficient} the authors propose the following algorithm to solve the following optimization problem:

\begin{equation}
\Pi_{\Delta_{N-1}}(\mathbf x_0)=\arginf\limits_{\mathbf x\in\Delta_{N-1}}\frac{1}{2}||\mathbf x-\mathbf x_0||_2^2.
\end{equation}


\begin{algorithm}
    \caption{Near Linear Time Projection Onto The Probability Simplex}
    \label{alg:projection}
    \begin{algorithmic}[1]
    \REQUIRE $\mathbf{z}\in\mathbb R^N$ \nonumber
    \STATE Sort $\mathbf z$ into $z_1\ge z_2\ge \ldots \ge z_N$ \label{line:sorting}
    \STATE Set $K\gets\max\left\{j=1,\ldots,N\biggr\rvert z_j-\frac{1}{j}\left(\sum\limits_{k=1}^jz_k-1\right)>0\right\}$
    \STATE Set $\theta=\frac{1}{K}\left(\sum\limits_{i=1}^Kz_i-1\right)$
    \STATE Set $w_i\gets(z_i-1)^+, \ \forall i\in1,\ldots,N$
    \RETURN $\mathbf w=(w_1,\ldots,w_N)$
    \end{algorithmic}
\end{algorithm}

The procedure is near linear since in Line \ref{line:sorting} we need to sort the input vector, that is known to be of $O(N\log N)$ complexity. Hence, Algorithm \ref{alg:projection} is a $\Theta(N\log N)$ procedure of projecting a $\mathbb R^N$ vector onto the probability simplex $\Delta_{N-1}$. Note that Algorithm \ref{alg:projection} can be refined to be of $\Theta(N)$ complexity if we avoid to sort the vector, that can be done as shown in \cite{duchi2008efficient}.
This shows that OGD is able to handle data streams that come at higher frequencies, \emph{e.g.}, the ones required by some specific financial applications \cite{abernethy2013adaptive}.

\section{Discussion on the Regret Bound}

In this section we will discuss some advantages of the OGD algorithm among the algorithms that bound the total regret $R_T^C$ defined in Definition \ref{def:totoal_regret}.

As discussed in Section \ref{sec:OLU}, the OLU algorithm is the only algorithm competing with OGD in terms of theoretical guarantees on the total regret.

Assuming to know \emph{a priori} the time horizon $T$ and under Assumption~\ref{ass:nojunk}, the authors of OLU provided the following guarantee on the total regret described in Theorem \ref{th:OLU}.
Notice that the OLU algorithm achieves a regret of $\mathcal{O}(\sqrt T)$ only if the transaction rate $\gamma \propto \frac{1}{\sqrt{T}}$, \emph{i.e.}, if the transaction rate decreases over time.
We observe that the first term of the r.h.s.~of Equation~\eqref{eq:OLU_regret} is the same as the corresponding one in Equation~\eqref{eq:OGD_T_known}: these terms correspond to the regret $R_T$.
Instead, if we focus on the second term of the r.h.s.~of Equation~\eqref{eq:OLU_regret} and we assume that $\gamma$ is constant over the investment horizon $T$, we would have a total regret of the order of $\mathcal{O}(T)$ for the OLU algorithm.
This does not happen to OGD, which, even under these assumptions, would provide a total regret of the order of $\mathcal{O}(\sqrt{T})$.
Conversely, if we assume $\gamma \propto \frac{1}{\sqrt{T}}$ as in~\cite{das2013online}, the last term in Equation~\eqref{eq:OGD_T_known} would have constant regret on the costs, \emph{i.e.}, $C_T(OGD)\le\frac{2 \epsilon_u N K}{\epsilon_l} = \mathcal{O}(1)$, compared to an order of $ \mathcal{O}(\sqrt{T})$ obtained by OLU, which makes OGD strictly better then OLU in terms of total regret bound.



